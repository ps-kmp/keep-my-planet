\documentclass[a4paper,11pt]{article}
\usepackage[utf8]{inputenc}
\usepackage[portuguese]{babel}
\usepackage[T1]{fontenc}
\usepackage{lmodern}
\usepackage{geometry}
\usepackage{hyperref}
\usepackage{enumitem}
\geometry{margin=2.5cm}
\usepackage{mwe}
\usepackage{etoolbox}
\patchcmd{\thebibliography}{\section*{\refname}}{}{}{}
\usepackage{float}
\usepackage[backend=biber, style=numeric, date=iso]{biblatex}


\begin{document}

\begin{titlepage}
    \centering
    \includegraphics[width=0.30\textwidth]{logo-isel.png}\par\vspace{1cm}
    {\LARGE \textsc{Instituto Superior de Engenharia de Lisboa}\par}
    \vspace{1cm}
    {\large \textsc{Licenciatura em Engenharia Informática e Computadores}\par}
    \vspace{1.5cm}
    {\huge\bfseries KeepMyPlanet: Sistema Kotlin Multiplatform para Limpeza Ambiental\par}
    \vspace{2cm}
    {\Large\itshape Diogo Ribeiro\par}
    {\large Telefone: +351 911 889 669 \par}
    {\large E-mail: a47207@alunos.isel.pt\par}
    \vspace{0.5cm}
    {\Large\itshape Rafael Pegacho\par}
    {\large Telefone: +351 913 464 971 \par}
    {\large E-mail: a49423@alunos.isel.pt\par}
    \vfill
    Orientador\par
    Eng. Pedro Pereira\par
    E-mail: pedro.pereira@isel.pt\par
    \vfill
    {\large \today\par}
\end{titlepage}

\section{Introdução}
\subsection{Contextualização}
Portugal e o mundo enfrentam um desafio ambiental significativo no que diz respeito à gestão de resíduos e poluição. De acordo com a Agência Portuguesa do Ambiente, no ano de 2023, a produção de resíduos urbanos em Portugal foi de 5,3 milhões de toneladas, registando um aumento de 0,28\% em relação ao ano anterior [1]. Ainda mais alarmante é o facto de que apenas 32\% destes resíduos são preparados para reutilização e reciclagem, valor consideravelmente inferior à meta europeia de 55\% estabelecida para 2025 [1].

A nível mundial, são geradas anualmente cerca de 2,24 mil milhões de toneladas de resíduos urbanos com projeções de aumento em 73\%, estimando-se um total de 3,88 mil milhões até 2050 [2]. A poluição residual não afeta apenas as áreas onde são depositadas, mas também contamina espaços naturais, urbanos e zonas costeiras.
Nas zonas costeiras, aproximadamente 80\% do lixo marinho é composto por plásticos [3].

Apesar da crescente consciencialização ambiental e do interesse da população em participar em ações de limpeza, existe uma lacuna significativa nas ferramentas disponíveis para identificar zonas poluídas e coordenar esforços de limpeza de forma eficaz. Atualmente, a organização de iniciativas de limpeza ambiental enfrenta desafios como a dispersão de informação, a dificuldade de coordenação entre voluntários e a falta de mecanismos para monitorizar o progresso das ações realizadas.

\subsection{Solução}
É neste contexto que surge o \textit{\textbf{KeepMyPlanet}}, um sistema multiplataforma desenvolvido com \textit{Kotlin Multiplatform (KMP)} [4] que permite identificar e mapear zonas poluídas, bem como organizar e participar em eventos comunitários de limpeza. O sistema proporciona uma interface interativa onde os voluntários podem assinalar áreas como poluídas, partilhar fotografias e descrições das condições encontradas, e criar ou aderir a iniciativas de limpeza organizadas para essas zonas.
\textit{\textbf{KeepMyPlanet}} é o mote de convergência para a ação comunitária e ambiental, abordando um problema real cada vez mais relevante e emergente.

\section{Análise}
O sistema será concebido com um conjunto de funcionalidades destinadas a facilitar a identificação, monitorização e coordenação das ações dos utilizadores voluntários, assegurando privacidade de dados e uma boa acessibilidade.

\subsection{Requisitos e Funcionalidades}
\subsubsection{Requisitos Funcionais}

\paragraph{Gestão de Utilizadores e Permissões -}
O sistema deve disponibilizar funcionalidades de registo e autenticação de utilizadores. Deve também ser capaz de diferenciar as permissões e funcionalidades de acordo com os diferentes perfis de utilizador, como \textit{Guest}, \textit{User}, \textit{Organizer} e \textit{Admin}.

\paragraph{Identificação e Sinalização de Zonas Poluídas -}
Deve ser possível que os utilizadores registados sinalizem áreas poluídas num mapa interativo, com a possibilidade de anexar fotografias e descrições detalhadas sobre o estado da área identificada.

\paragraph{Gestão de Eventos -}
O sistema deve permitir que utilizadores registados possam criar, editar e excluir eventos de limpeza. Cada evento está associado a uma zona previamente assinalada no mapa e corresponde a uma data e hora onde os seus participantes se unem para a limpeza dessa mesma zona. Deve ainda possibilitar a inscrição dos utilizadores nos eventos e a confirmação da sua participação. Para facilitar o controlo da presença, será implementado um sistema de validação, por exemplo, através da leitura de código QR, fornecido pelos organizadores, e lido pelos voluntários. O sistema também deverá disponibilizar estatísticas sobre o impacto de cada evento.

\paragraph{Comunicação e Colaboração -}
Será implementado um sistema de mensagens, associado a cada evento, para permitir a comunicação entre os participantes do mesmo. Além disso, os organizadores terão a capacidade de enviar atualizações e notificações para os utilizadores inscritos nos seus eventos, garantindo uma comunicação eficaz e atualizada.

\begin{figure}[H]
    \centering
    \includegraphics[width=0.6\textwidth]{Use-Cases-azul-uml.png}
    \caption{Diagrama \textit{Use Case} do sistema}
    \label{fig:usecase}
\end{figure}

\subsubsection{Requisitos Não Funcionais}

\paragraph{Maximizar o código partilhado usando KMP -}
O sistema deve ser desenvolvido aproveitando ao máximo as capacidades do \textit{\textbf{Kotlin Multiplatform}}, visando a partilha extensiva de código entre as diferentes plataformas suportadas. Esta abordagem é crucial para garantir consistência funcional, reduzir significativamente o tempo de desenvolvimento e facilitar a manutenção do sistema a longo prazo. A lógica da aplicação, gestão de estado, comunicação com APIs e persistência de dados devem ser implementadas na camada comum, permitindo que apenas os componentes específicos de UI (\textit{User Interface}) de cada plataforma sejam desenvolvidos separadamente, promovendo assim uma maior eficiência de recursos na experiência do utilizador em todos os dispositivos.

\paragraph{Desempenho e Escalabilidade -}
O sistema deve ser capaz de suportar um grande número de utilizadores simultâneos sem comprometer a experiência de utilização. As operações críticas, como a sinalização de zonas no mapa e a inscrição em eventos, não podem ser demoradas, garantindo que o sistema se mantenha eficiente mesmo em cenários de elevada afluência.

\paragraph{Segurança e Privacidade -}
A autenticação deve ser implementada de forma segura, garantindo que o acesso às funcionalidades exclusivas dos utilizadores registados seja protegido. Além disso, a proteção de dados dos utilizadores deve ser assegurada em conformidade com o Regulamento Geral sobre a Proteção de Dados, para garantir a privacidade e segurança das informações pessoais.

\paragraph{Acessibilidade -}
A interface do sistema deve ser intuitiva e fácil de utilizar, garantindo que diferentes perfis de utilizadores possam adotar a plataforma com facilidade. O sistema também deve ser compatível com diferentes dispositivos.


\subsection{Tecnologia}
O sistema será implementado utilizando \textit{\textbf{Kotlin Multiplatform (KMP)}} [4], uma tecnologia que possibilita a partilha de código entre várias plataformas, com foco na redução da duplicação de código facilitando a manutenção do sistema e a escrita de código. A arquitetura do sistema será baseada no padrão \textit{\textbf{Model-View-Viewmodel (MVVM)}}, permitindo uma separação clara entre a interface do utilizador e a lógica do sistema, promovendo a escalabilidade e a manutenção do código, para além de melhorar a testabilidade.
A integração com uma API externa de mapas representa um dos maiores desafios do projeto. Esta API será responsável por permitir a sinalização de zonas poluídas no mapa interativo, sendo essencial que a comunicação entre a aplicação e o serviço externo seja eficiente e fiável. 
Para efetuar essa integração, será usada a \textit{framework \textbf{Ktor}} [5], especializada na realização de pedidos HTTP. Será responsável por garantir que a comunicação com a API é efetuada de forma assíncrona, otimizada e segura, garantindo uma correta gestão de dados e erros. Esta abordagem assegura uma interação fluida com o serviço de mapas, fundamental para o desempenho global da aplicação.

\section{Planeamento}
\begin{figure}[H]
    \centering
    \includegraphics[width=0.6\textwidth]{cronograma.png}
    \caption{Planeamento do projeto}
    \label{fig:usecase}
\end{figure}

\section{Referências} 
\begin{thebibliography}{9}

\bibitem{APAmbiente2023} Agência Portuguesa do Ambiente (APA). (2023). Relatório Anual de Resíduos Urbanos 2023. Disponível em: \url{https://apambiente.pt/sites/default/files/_Residuos/Producao_Gest%C3%A3o_Residuos/Dados%20RU/2023/raru_2023.pdf} Acedido em 10/03/2025.

\bibitem{WorldBank2023} World Bank Group. (2022). Solid Waste Management. Disponível em: \url{https://www.worldbank.org/en/topic/urbandevelopment/brief/solid-waste-management} Acedido em 10/03/2025.

\bibitem{IUCN2023} International Union for Conservation of Nature (IUCN). (2024). Plastic Pollution. Disponível em: \url{https://iucn.org/resources/issues-brief/plastic-pollution} Acedido em 10/03/2025.

\bibitem{KotlinMultiplatform} Kotlin. (n.d.). Introduction to Kotlin Multiplatform. Disponível em: \url{https://kotlinlang.org/docs/multiplatform-intro.html} Acedido em 10/03/2025.

\bibitem{Ktor} Ktor. (n.d.). Welcome to Ktor. Disponível em: \url{https://ktor.io/docs/welcome.html} Acedido em 10/03/2025.

\end{thebibliography}


\end{document}